%% abtex2-modelo-trabalho-academico.tex, v-1.9.6 laurocesar
%% Copyright 2012-2016 by abnTeX2 group at http://www.abntex.net.br/ 
%%
%% This work may be distributed and/or modified under the
%% conditions of the LaTeX Project Public License, either version 1.3
%% of this license or (at your option) any later version.
%% The latest version of this license is in
%%   http://www.latex-project.org/lppl.txt
%% and version 1.3 or later is part of all distributions of LaTeX
%% version 2005/12/01 or later.
%%
%% This work has the LPPL maintenance status `maintained'.
%% 
%% The Current Maintainer of this work is the abnTeX2 team, led
%% by Lauro César Araujo. Further information are available on 
%% http://www.abntex.net.br/
%%
%% This work consists of the files abntex2-modelo-trabalho-academico.tex,
%% abntex2-modelo-include-comandos and abntex2-modelo-references.bib
%%

% ------------------------------------------------------------------------
% ------------------------------------------------------------------------
% abnTeX2: Modelo de Trabalho Academico (tese de doutorado, dissertacao de
% mestrado e trabalhos monograficos em geral) em conformidade com 
% ABNT NBR 14724:2011: Informacao e documentacao - Trabalhos academicos -
% Apresentacao
% ------------------------------------------------------------------------
% ------------------------------------------------------------------------

\documentclass[
	% -- opções da classe memoir --
	12pt,				% tamanho da fonte
	% openright,			% capítulos começam em pág ímpar (insere página vazia caso preciso)
	oneside,			% twoside para impressão em recto e verso. Oposto a oneside
	a4paper,			% tamanho do papel. 
	% -- opções da classe abntex2 --
	chapter=TITLE,		% títulos de capítulos convertidos em letras maiúsculas
	section=TITLE,		% títulos de seções convertidos em letras maiúsculas
	%subsection=TITLE,	% títulos de subseções convertidos em letras maiúsculas
	%subsubsection=TITLE,% títulos de subsubseções convertidos em letras maiúsculas
	% -- opções do pacote babel --
	english,			% idioma adicional para hifenização
	%french,				% idioma adicional para hifenização
	%spanish,			% idioma adicional para hifenização
	brazil				% o último idioma é o principal do documento
	]{abntex2}

% ---
% Pacotes básicos 
% ---
\usepackage{lmodern}			% Usa a fonte Latin Modern			
\usepackage[T1]{fontenc}		% Selecao de codigos de fonte.
\usepackage[utf8]{inputenc}		% Codificacao do documento (conversão automática dos acentos)
\usepackage{lastpage}			% Usado pela Ficha catalográfica
\usepackage{indentfirst}		% Indenta o primeiro parágrafo de cada seção.
\usepackage{color}				% Controle das cores
\usepackage{graphicx}			% Inclusão de gráficos
\usepackage{microtype} 			% para melhorias de justificação
% ---
		
% ---
% Pacotes adicionais, usados apenas no âmbito do Modelo Canônico do abnteX2
% ---
\usepackage{lipsum}				% para geração de dummy text
% ---

%% Adiciona as alterações do ABNTeX-IFPI
\usepackage{config/abntex-ifms}	% Modificações do ABNTeX para o IFPI
% \usepackage{config/tikz-uml}	    % Pacote Tikz UML para criar UML no LaTeX

% --- 
% CONFIGURAÇÕES DE PACOTES
% --- 

% Configuracoes do arquivo PDF gerado
% ---
% Configurações de aparência do PDF final

% informações do PDF
\makeatletter
\hypersetup{
		%pagebackref=true,
		pdftitle={\@title}, 
		pdfauthor={\@author},
		pdfsubject={\imprimirpreambulo},
		pdfcreator={LaTeX with abnTeX2},
		pdfkeywords={abnt}{latex}{abntex}{abntex2}{trabalho acadêmico}, 
		colorlinks=true,			% Visual dos Links: false = caixas; true = colorido
		linkcolor=cor-link,		% Cor dos Links Internos (preto)
		citecolor=cor-link,		% Cor de Links para Bibliografia (preto)
		filecolor=cor-link,		% Cor para Links a Arquivos (preto)
		urlcolor=cor-link,		% Cor para Links a URLs (preto)
		bookmarksdepth=4
}
\makeatother
% --- 

% Tipografia
% Abra este arquivo e selecione uma das opções de fonte nele. A padrão é Times.
%% Tipografia / Fontes
%% AVISO: Todas essas fontes são *bastante semelhantes* aos nomes com as quais as descrevo. Entenda: são iguais, só que oficialmente com outro nome.

%% %%%%%%%%%%%%%%%%%%%%%%%%%%%%%%%%%%%%%%%%%%%%%%%%%%%%% %%
%% Comente todas as outras fontes que você não vai usar! %%
%% %%%%%%%%%%%%%%%%%%%%%%%%%%%%%%%%%%%%%%%%%%%%%%%%%%%%% %%

%% Latin Modern (fonte padrão do LaTeX, Computer Modern, mas com suporte a caracteres acentuados)
%% Considerada a mais clássica e bonita
%\usepackage{lmodern}



%% Times
%% Considerada a mais confortável de ler quando impresso
% \usepackage{mathptmx}

%% Variação da mesma fonte, com minúsculas diferenças entre uma e outra (coisas bastante técnicas como kerning, aliasing e afins) - Essa tem revisões frequentes
%\usepackage{newtxtext} \usepackage{newtxmath}



%% Arial
%% Considerada mais confortável de ler num computador
%\usepackage{helvet} \renewcommand{\familydefault}{\sfdefault}



%% Palatino
%% Uma opção mais elegante à Times
%\usepackage{newpxtext}



%% Kepler
%% Variação evoluída da Palatino, com várias pequenas diferenças e refinamentos
%\usepackage{kpfonts}



%% Libertine
%% Uma fonte estilo Serif comum no Linux
%\usepackage{libertine} %\usepackage[libertine]{newtxmath}

% Metadados
%% %%%%%%%%%%%%%%%%%%%%%%%%%%%%%%%%%%%%%%%%%%%%%%%% %%
%% Metadados do trabalho
%% Informações de dados para CAPA e FOLHA DE ROSTO
%% AVISO: Todos esses dados serão automaticamente convertidos para caixa alta onde necessário
%% %%%%%%%%%%%%%%%%%%%%%%%%%%%%%%%%%%%%%%%%%%%%%%%% %%

\titulo{Título Trabalho de conclusão de curso}

\autor{Marcos Azevedo Silva}
\local{Naviraí}
\data{2018}

%% "M\textsuperscript{e}." = Abreviação oficial para "Mestre"
\orientador{Prof. M\textsuperscript{e}. Nome do Orientador}
%\coorientador{Clara Castro Cardoso}

\instituicao{%
  INSTITUTO FEDERAL DE EDUCAÇÃO, CIÊNCIA E TECNOLOGIA DO MATO GROSSO DO SUL
  \par
  CAMPUS NAVIRAÍ
  \par
  TECNOLOGIA EM ANÁLISE E DESENVOLVIMENTO DE SISTEMAS}
  
%% Tipo de Trabalho
%% - Monografia
%% - Tese (Mestrado)
%% - Tese (Doutorado)
%% - Relatório técnico
\tipotrabalho{Monografia}

% O preambulo deve conter o tipo do trabalho, o objetivo, 
% o nome da instituição e a área de concentração 
\preambulo{Projeto apresentado à Banca Examinadora como requisito para aprovação na disciplina de Projeto Integrador I do Curso Superior de Tecnologia em Análise e Desenvolvimento de Sistemas do Instituto Federal de Educação, Ciência e Tecnologia do Mato Grosso do Sul.}

%% Primeiro membro da banca examinadora
\membroum{Prof. M\textsuperscript{e}. Nome do Membro 1}
% {Instituto Federal de Educação, Ciência e Tecnologia do Mato Grosso do Sul - IFMS}

%% Segundo membro da banca examinadora
\membrodois{Prof. M\textsuperscript{e}. Nome do Membro 2}

%% Terceiro membro da banca examinadora - Se houver
% \membrotres{Prof. Dr. Ney Inga de Oliveira Nome do Membro 3}

%% Data da apresentação do trabalho
%% Se não souber a data da apresentação, utilize \underline{\hspace{3.5cm}}
%% Isso cria um sublinhado de 3.5cm, onde você pode escrever a data depois!
%\dataapresentacao{04 de Abril de 2017}
%\dataapresentacao{\underline{\hspace{3.5cm}}}

% Configuração das citações
%% Configuração das Citações


\usepackage[brazilian,hyperpageref]{backref}	 % Paginas com as citações na bibl

% \usepackage[num]{abntex2cite}			% Citações numéricas
\usepackage[alf]{abntex2cite}	% Citações "AUTOR, ano" - Padrão ABNT


%% Colocar entre parênteses ou colchetes?
%% Padrão: Parênteses
%% * Fica mais agradável usar colchetes quando se usa citações numéricas
%% \citebrackets[] % Comente essa linha e o documento usará parênteses


%% Configura o "Citado nas Páginas ..." nas referências
%% Não mexa nesse:
\renewcommand{\backref}{}

%% Esse é o texto do "Citado nas páginas ..."
\renewcommand*{\backrefalt}[4]{
	\ifcase #1
		Nenhuma citação no texto.
	\or
		Citado na página #2.
	\else
		Citado #1 vezes nas páginas #2.
	\fi}

% Cores
%% Cores do Documento

%% Cor dos Links do PDF
%% Usando preta você "esconde" os links
% \definecolor{cor-link}{RGB}{0,0,0}

%% Usando azul os links ficam visíveis (ruim para impressão)
\definecolor{cor-link}{RGB}{41,5,195}
%\definecolor{cor-link}{RGB}{8,40,75}



%% Cor para os quadros
%% Dê preferência a cores escuras.
%% Boa referência para cores: https://material.io/guidelines/style/color.html#color-color-palette
\definecolor{cor-quadro}{RGB}{5,28,63}		% Azul Escuro

% Espacamentos
% --- 
% Espaçamentos entre linhas e parágrafos 
% --- 

% O tamanho do parágrafo é dado por:
\setlength{\parindent}{1.3cm}

% Controle do espaçamento entre um parágrafo e outro:
\setlength{\parskip}{0.2cm}  % tente também \onelineskip